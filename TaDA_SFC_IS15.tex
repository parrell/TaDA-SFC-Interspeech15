\documentclass[a4paper]{article}

\usepackage{INTERSPEECH2015}

\usepackage{graphicx}
\usepackage{amssymb,amsmath,bm}
\usepackage{textcomp}

\def\vec#1{\ensuremath{\bm{{#1}}}}
\def\mat#1{\vec{#1}}


\sloppy % better line breaks
\ninept

\title{A task dynamics model of speech articulation with state feedback control}

%%%%%%%%%%%%%%%%%%%%%%%%%%%%%%%%%%%%%%%%%%%%%%%%%%%%%%%%%%%%%%%%%%%%%%%%%%
%% If multiple authors, uncomment and edit the lines shown below.       %%
%% Note that each line must be emphasized {\em } by itself.             %%
%% (by Stephen Martucci, author of spconf.sty).                         %%
%%%%%%%%%%%%%%%%%%%%%%%%%%%%%%%%%%%%%%%%%%%%%%%%%%%%%%%%%%%%%%%%%%%%%%%%%%
%\makeatletter
%\def\name#1{\gdef\@name{#1\\}}
%\makeatother
%\name{{\em Firstname1 Lastname1, Firstname2 Lastname2, Firstname3 Lastname3,}\\
%      {\em Firstname4 Lastname4, Firstname5 Lastname5, Firstname6 Lastname6,
%      Firstname7 Lastname7}}
%%%%%%%%%%%%%%% End of required multiple authors changes %%%%%%%%%%%%%%%%%

\makeatletter
\def\name#1{\gdef\@name{#1\\}}
\makeatother \name{{\em Vikram Ramanarayanan$^{1,\dagger}$, Ben Parrell$^{2,\dagger}$, John Houde$^3$, Srikantan Nagarajan$^4$ and Louis Goldstein$^5$ }}

\address{$^1$Educational Testing Service R\&D \\
  $^2$Department of Psychology, UC Berkeley\\
  $^3$Department of Otolaryngology, UCSF \\
  $^4$Department of Radiology, UCSF \\
  $^5$Department of Linguistics, USC \\
  {\small \tt vramanarayanan@ets.org,parrell@gmail.com}
}

%\twoauthors{Karen Sp\"{a}rck Jones.}{Department of Speech and Hearing \\
%  Brittania University, Ambridge, Voiceland \\
%  {\small \tt Karen@sh.brittania.edu} }
%  {Rose Tyler}{Department of Linguistics \\
%  University of Speechcity, Speechland \\
%  {\small \tt RTyler@ling.speech.edu} }

%
\begin{document}

  \maketitle
  %
  \begin{abstract}

  We present a task dynamics model of speech articulation that explicitly incorporates a framework for control based on instantaneous and delayed auditory state feedback. We do this by combining two existing well-known theories in the speech production literature -- the Task Dynamics model of speech production \cite{SALTZMAN89} and the State Feedback Control model of speech motor control \cite{HOUDE12}. We demonstrate the effectiveness of the model by simulating a simple perturbation study. 
  \let\thefootnote\relax\footnotetext{$^\dagger$The first two authors contributed equally to the work.} 
  \end{abstract}
  \noindent{\bf Index Terms}: task dynamics, articulatory phonology, state feedback control


  \section{Introduction}


  \section{Acknowledgements}
  


  \newpage
  \eightpt
  \bibliographystyle{IEEEtran}
  \bibliography{TaDA_SFC_IS15}

\end{document}
